\title{数学II 中間試験}
\date{}
\maketitle

\vspace*{-2cm}
\begin{table}[!h]
	\begin{flushright}
		\begin{tabular}{rc}
			\vspace*{0.1in} & \usdate\today\\
			学籍番号 & \vspace{0.1in} \\
			名前 & \vspace{0.1in} \\
		\end{tabular}
	\end{flushright}
\end{table}

\begin{itembox}[c]{解答上の注意}
	\begin{itemize}
		\item 設問に指定がない場合は,計算過程も含めて,記述してください.
		\item 穴埋め問題は,解答だけを,記入してください.
		\item 試験時間は,90分です.
		\item 大問7と大問8は,選択問題です.どちらか1問を選んで解答してください.(両方解答しても構いません.得点を上乗せします.)
	\end{itemize}
\end{itembox}

\newpage
\begin{questions}
	\question
	次の問いに答えなさい.
	\begin{parts}
		\part
		以下は,三角関数の値を表にしたものである.空欄を埋めよ.
		\begin{table}[!h]
			%\centering
			\begin{tabular}{c|c|c|c|c|c}
				$\theta$ & $0$ & $\pi/6$ & $\pi/4$ & $\pi/3$ & $\pi/2$ \\
				\hline
				$\sin\theta$ & \mySpace & & & \mySpace & \\
				\hline
				$\cos\theta$ & & \mySpace & & & \mySpace \\
				\hline
				$\tan\theta$ & & & \mySpace & & \\
			\end{tabular}
		\end{table}
		\begin{table}[!h]
			%\centering
			\begin{tabular}{c|c|c|c|c}
				$\theta$ & $2\pi/3$ & $3\pi/4$ & $5\pi/6$ & $\pi$ \\
				\hline
				$\sin\theta$ & \mySpace & & & \mySpace \\
				\hline
				$\cos\theta$ & & \mySpace & & \\
				\hline
				$\tan\theta$ & & & \mySpace & \\
			\end{tabular}
		\end{table}
	
		\part
		以下の値を求めよ.
		\begin{subparts}
			\subpart $\displaystyle \arcsin\frac{1}{2}$
			\answerline
			\subpart $\displaystyle \arccos\left(-\frac{1}{\sqrt{2}}\right)$
			\answerline
			\subpart $\displaystyle \arctan\sqrt{3}$
			\answerline
		\end{subparts}
	\end{parts}

	\newpage
	\question
	次の問いに答えなさい.
	\begin{parts}
		\part
		定義に従って$f(x) = 1/x$の微分を求めなさい.
		\makeemptybox{7cm}
		
		\part
		$f'(2)$の値を求めなさい
		\answerline
		
		\part
		点$x = 2$における接線の方程式を求めなさい.
		\makeemptybox{7cm}
	\end{parts}

	\newpage
	\question
	関数$f(x) = |x|$は,点$x = 0$において微分不可能である.その理由を記述しなさい.
	\makeemptybox{18cm}
	
	\newpage
	\question
	次の問いに答えなさい.
	\begin{parts}
		\part
		三角関数の加法定理を記述しなさい.
		\begin{subparts}
			\subpart
			$\sin\left(\alpha+\beta\right)$ = \mySpace
			\subpart
			$\cos\left(\alpha+\beta\right)$ = \mySpace	
		\end{subparts}
	
		\part
		三角関数の加法定理を用いて,以下の和積公式を導出しなさい.
		\[
			\sin A - \sin B = 2\cos\left(\frac{A+B}{2}\right)\sin\left(\frac{A-B}{2}\right)
		\]
		\makeemptybox{14cm}
		
		\part
		定義に従って$f(x) = \sin x$の微分を求めなさい.必要に応じて$\displaystyle \lim\limits_{x \to 0}\frac{\sin x}{x} = 1$を用いてもよい.
		\makeemptybox{18cm}
	\end{parts}

	\newpage
	\question
	%逆関数の微分を用いて,$f(x) = a^x$の微分を求めなさい.
	%\makeemptybox{18cm}
	\setlength\answerlinelength{3in}
	次の関数をそれぞれ微分しなさい.
	\begin{parts}
		\part
		$\displaystyle e^{(2-3x)}$
		\answerline
		
		\part
		$\displaystyle (x^2-x)\ln x$
		\answerline
		
		\part
		$\displaystyle \log_{2}x$
		\answerline
		
		\part
		$\displaystyle x\cos x - \sin x$
		\answerline
	\end{parts}
	\setlength\answerlinelength{1in}
	
	\newpage
	\question
	次の問いに答えなさい.
	\begin{parts}
		\part
		ライプニッツの公式を用いて,$f(x) = x^22^x$の$n$階導関数を求めなさい.
		\makeemptybox{14.5cm}
		
		\part
		$f(x)$の1階導関数$f^{(1)}$,2階導関数$f^{(2)}$を求めなさい.
		\makeemptybox{5cm}
		
		%\part
		%関数の増減表を作り,グラフの概形を描きなさい.
		%\makeemptybox{10cm}
	\end{parts}

	\newpage
	\question
	関数$\displaystyle f(x) = e^{-x^2}$の極値を調べ,グラフの概形を描きなさい.
	
	\newpage
	\question
	関数$\displaystyle f(x) = x^4-6x^2-8x+6$の極値を調べ,グラフの概形を描きなさい.また,閉区間$[0, 3]$における最大値・最小値を求めなさい.
\end{questions}