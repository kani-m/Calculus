\title{数学II 期末試験}
\date{}
\maketitle

\vspace*{-2cm}
\begin{table}[!h]
	\begin{flushright}
		\begin{tabular}{rc}
			\vspace*{0.1in} & \usdate\today\\
			学籍番号 & \vspace{0.1in} \\
			名前 & \vspace{0.1in} \\
		\end{tabular}
	\end{flushright}
\end{table}

\begin{itembox}[c]{解答上の注意}
	\begin{itemize}
		\item 設問に指定がない場合は,計算過程も含めて,記述してください.
		\item 穴埋め問題は,解答だけを,記入してください.また,余白に計算過程が記述されている場合,部分点を考慮する可能性があります.
		\item 試験時間は,90分です.
		%\item 大問7と大問8は,選択問題です.どちらか1問を選んで解答してください.(両方解答しても構いません.得点を上乗せします.)
	\end{itemize}
\end{itembox}

\newpage
\begin{questions}
	\question
	以下は,関数$f(x)$について,その導関数$f'(x)$と不定積分$\int f(x)\,\dif x$を表にしたものである.空欄を埋めよ.(ただし,積分定数は省略してよい.)
	\begin{table}[!h]
		\centering
		\begin{tabular}{c|c|c}
			$f'(x)$ & $f(x)$ & $\int f(x)\,\dif x$ \\
			\hline
			\mySpace & $x^a\quad(a \neq -1)$ & \\
			\hline
			& $\displaystyle \frac{1}{x}$ & \mySpace \\
			\hline
			\mySpace & $\sin x$ & \\
			\hline
			& $\cos x$ & \mySpace \\
			\hline
			\mySpace & $\tan x$ & $-\ln\left|\cos x\right|$ \\
			\hline
			& $e^x$ & \mySpace \\
			\hline
			\mySpace & $\ln x$ & \\
		\end{tabular}
	\end{table}
	
	\newpage
	\question
	\setlength\answerlinelength{3in}
	以下の関数をそれぞれ微分しなさい.
	\begin{parts}
		\part
		$\displaystyle x^3 + 3x^2 -\frac{2}{x}$
		\answerline
		
		\part
		$\displaystyle \sqrt{1 - x^2}$
		\answerline
		
		\part
		$\displaystyle e^{2x - 1}$
		\answerline
		
		\part
		$\displaystyle x\sin x - \cos x$
		\answerline
		
		%\part
		%$\displaystyle \frac{1}{1 + \cos x}$
		%\answerline
		
		\part
		$\displaystyle \left(\ln x\right)^2$
		\answerline
	\end{parts}

	\newpage
	\question
	以下の曲線$y = f(x)$について,指示された値に対応する点における接線の方程式を求めよ.
	\begin{parts}
		\part
		$\displaystyle f(x) = 3x^2\quad(x = 2)$
		\answerline
		
		\part
		$\displaystyle f(x) = \sqrt{x}\quad(x = 4)$
		\answerline
		
		%\part
		%$\displaystyle f(x) = \ln(2x - 1)\quad(x = 1)$
		%\answerline
		
		\part
		$\displaystyle f(x) = xe^x\quad(x=0)$
		\answerline
	\end{parts}
	\setlength\answerlinelength{1in}
	
	\newpage
	\question
	関数$\displaystyle f(x) = x^4 + 4x^3 - 16x - 3$の極値を調べ,グラフの概形を描きなさい.
	
	\newpage
	\question
	\setlength\answerlinelength{3in}
	以下の関数をそれぞれ不定積分しなさい.
	\begin{parts}
		\part
		$\displaystyle x^3 + 3x^2 -\frac{2}{x}$
		\answerline
				
		\part
		$\displaystyle \frac{3}{2\sqrt{x}}$
		\answerline

		\part
		$\displaystyle 2e^{2x} + 3^x$
		\answerline
		
		\part
		$\displaystyle 3\sin 3x - \cos(x - 2)$
		\answerline
	\end{parts}

	\newpage
	\question
	以下の関数をそれぞれ不定積分しなさい.
	\begin{parts}
		%\part
		%$\displaystyle x(1 - x)^5$
		%\answerline

		\part
		$\displaystyle x\sqrt{x + 1}$
		\answerline

		\part
		$\displaystyle x\cos x$
		\answerline
		
		\part
		$\displaystyle xe^x$
		\answerline
	\end{parts}
	
	\newpage
	\question
	以下の定積分の値をそれぞれ求めなさい.
	\begin{parts}
		\part
		$\displaystyle \int_1^2 \left(2x + \frac{1}{x}\right)\dif x$
		\answerline
		
		\part
		$\displaystyle \int_0^2 x(2 - x)^3\,\dif x$
		\answerline
		
		\part
		$\displaystyle \int_0^{\frac{\pi}{2}}\sin^2x\cos x\,\dif x$
		\answerline
		
		\part
		$\displaystyle \int_1^3 x\ln x\,\dif x$
		\answerline
	\end{parts}
	\setlength\answerlinelength{1in}

	\newpage
	\question
	関数$\displaystyle f(x) = x^3 + x^2 - 4x$について,点$x = 1$における接線の方程式を求めなさい.また,その接線と曲線$y = f(x)$とで囲まれた面積を求めなさい.
	
	\newpage
	\question
	関数$\displaystyle f(x) = x^3 - x^2 - x + 1$について,方程式$f(x) = 0$の解を求めなさい.また,曲線$y = f(x)$と$x$軸とで囲まれた面積を求めなさい.
\end{questions}