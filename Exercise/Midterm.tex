\section*{中間試験参考問題}
\begin{table}[!t]
	\begin{flushright}
		\begin{tabular}{rc}
			& \usdate\today \\
			学籍番号 & \\
			名前 & \\
		\end{tabular}
	\end{flushright}
\end{table}
別途,ノートかルーズリーフか白紙の計算用紙上に,計算過程も含めて,解いてください.

\begin{question}
	以下は,三角関数の値を表にしたものである.空欄を埋めよ.
\end{question}
\begin{table}[!h]
	\centering
	\begin{tabular}{c|ccccccccc}
		$\theta$ & $0$ & $\pi/6$ & $\pi/4$ & $\pi/3$ & $\pi/2$ & $2\pi/3$ & $3\pi/4$ & $5\pi/6$ & $\pi$ \\
		\hline
		$\sin\theta$ & & & & & & & & & \\
		$\cos\theta$ & & & & & & & & & \\
		$\tan\theta$ & & & & & & & & & \\
	\end{tabular}
\end{table}

\begin{question}
	以下の三角関数に関する各種定理・公式を書きなさい.
	\begin{enumerate}[itemsep=2ex, label*=(\arabic*)]
		\item 三角関数の加法定理
		\item 三角関数の和積公式
		\item 三角関数の積和公式
	\end{enumerate}
\end{question}

\begin{question}
	以下の値を求めよ.
	\begin{enumerate}[itemsep=2ex, label*=(\arabic*)]
		\item $\displaystyle \arcsin\frac{1}{2}$
		\item $\displaystyle \arccos\left(-\frac{1}{\sqrt{2}}\right)$
		\item $\displaystyle \arctan\sqrt{3}$
	\end{enumerate}
\end{question}

\begin{question}
	定義に従って以下を微分せよ.
	\begin{enumerate}[itemsep=2ex, label*=(\arabic*)]
		\item $\displaystyle 2x^2 - x$
		\item $\displaystyle x^3$
		\item $\displaystyle \frac{1}{x}$
	\end{enumerate}
	また,それぞれの関数について,点$x = 2$における微分係数を求めなさい.
\end{question}

\begin{question}
	関数$f(x) = |x|$について,点$x = 0$における微分係数を調べなさい.
\end{question}

\begin{question}
	以下の関数を微分せよ.
	\begin{tabenum}[(1)]
		\item $\displaystyle x^4+3x^2-5x+7$
		\item $\displaystyle \frac{1}{x^2}$
		\item $\displaystyle x^{2/3}$
		\item $\displaystyle (x^2+1)(2x^3+3x)$ \\[2ex]
		\item $\displaystyle (3x+1)(2-3\sqrt{x})$
		\item $\displaystyle \frac{1}{4x+3}$
		\item $\displaystyle (1+x^4)^6$
		\item $\displaystyle 3^x$ \\[2ex]
		\item $\displaystyle e^x(2-3x)$
		\item $\displaystyle e^{3x+7}$
		\item $\displaystyle (x^2-x)\ln x$
		\item $\displaystyle \log_{2}x$ \\[2ex]
		\item $\displaystyle (1+e^x)\ln x$
		\item $\displaystyle 2^x\log_{2} x$
		\item $\displaystyle x\cos x$
		\item $\displaystyle \sin x\ln x$ \\[2ex]
		\item $\displaystyle 2x\tan x$
		\item $\displaystyle \sqrt{1 + \cos x}$
		\item $\displaystyle \sin(1+x^2)$
		\item $\displaystyle \cos(1-x^2)$ \\[2ex]
		\item $\displaystyle \tan^5x$
		\item $\displaystyle \arcsin x$
		\item $\displaystyle \arcsin\frac{x}{2}$
		\item $\displaystyle \arccos x^2$ \\[2ex]
	\end{tabenum}
\end{question}

\begin{question}
	逆関数の微分を用いて,$f(x) = a^x$の微分を求めよ.
\end{question}

\begin{question}
	ライプニッツの公式を用いて,次の関数の$n$階導関数を求めよ.
	\begin{enumerate}[itemsep=2ex, label*=(\arabic*)]
		\item $\displaystyle f(x) = x^32^x$
		\item $\displaystyle f(x) = e^x(2-3x)$
		\item $\displaystyle f(x) = x\cos x$
	\end{enumerate}
\end{question}

\begin{question}
	次の関数の,指定された点における接線の方程式を求めよ.
	\begin{enumerate}[itemsep=2ex, label*=(\arabic*)]
		\item $\displaystyle f(x) = x^3\quad(x = 3)$
		\item $\displaystyle f(x) = \sqrt{x}\quad(x = 2)$
		\item $\displaystyle f(x) = \cos x\quad(x = \pi/4)$
		\item $\displaystyle f(x) = \log_{3} x\quad(x = 9)$
	\end{enumerate}
\end{question}

\begin{question}
	次の関数の,極値を調べ,グラフの概形を書きなさい.
	\begin{enumerate}[itemsep=2ex, label*=(\arabic*)]
		\item $\displaystyle f(x) = x^3-3x$
		\item $\displaystyle f(x) = (x-1)^{2/3}$
		\item $\displaystyle f(x) = xe^{-x}$
		\item $\displaystyle f(x) = 24x - 6x^2 - 8x^3 + 3x^4$
	\end{enumerate}
\end{question}

\begin{question}
	次の関数の,指定された区間における最大・最小を求めよ.
	\begin{enumerate}[itemsep=2ex, label*=(\arabic*)]
		\item $\displaystyle f(x) = x^3-3x\quad x \in \left[-2, 3\right]$
		\item $\displaystyle f(x) = (x-1)^{2/3}\quad x \in \left[0, 3\right]$
		\item $\displaystyle f(x) = xe^{-x}\quad x \in \left[-1, 1\right]$
	\end{enumerate}
\end{question}