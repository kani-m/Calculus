% % % Document Type
% http://www.biwako.shiga-u.ac.jp/sensei/kumazawa/tex/book.html
% http://ichiro-maruta.blogspot.jp/2013/03/latex.html
\RequirePackage[l2tabu, orthodox]{nag}
% http://www-math.mit.edu/~psh/exam/examdoc.pdf
\documentclass[12pt, a4j, openany, uplatex, dvipdfmx]{exam}

% % % Paper size
%\usepackage[top=1truein, bottom=1truein, left=1.5truein, right=1truein]{geometry} % 用紙サイズの設定のため

% % % Necessary
\usepackage{amssymb, amsmath, amsthm, latexsym} % For mathematics [Before hyperref]
\usepackage{empheq} % For mathematics
\usepackage[dvipdfmx,setpagesize=false]{hyperref} % For inserting hyperlink [Before graphicx]
\usepackage{pxjahyper} % https://texwiki.texjp.org/?hyperref
\usepackage[dvipdfmx]{graphicx} % For inserting figure
\usepackage[labelformat=simple]{subcaption} % For caption of figures and tables
%\renewcommand\thesubfigure{(\alph{subfigure})}
%\renewcommand\thesubtable{(\alph{subtable})}
\usepackage[usenames,dvipdfmx]{color} % For coloring
\usepackage[shortlabels]{enumitem} % For useful enumerate and itemize environment
\usepackage{comment} % For commenting multiline
\usepackage{cite} % For citing references [After hyperref]
\usepackage{url} % For inserting URL
\usepackage{acro} % For listing abbreviations and symbols
\usepackage{makeidx} % For making index

% % % Convenient for graph
\usepackage{tikz-cd} %load package after color.sty & soul.sty

% % % Convenient for tables and emphasis
\usepackage{tabularx} % For flexible table
\usepackage{multirow} % For combining multi rows for table
%\usepackage{colortbl} % For coloring table
\usepackage{longtable} % For long table across multiple pages
%\usepackage[normalem]{ulem} % For flexible underline etc.
\usepackage{ascmac} % For framing multiline -> bxascmac に変更したほうがよい?

% % % 文書の見た目のためのパッケージ
\usepackage{setspace} % For single space or double space
\usepackage{afterpage} % For inseting newpage 
\usepackage{datetime} % For date style of title
%\usepackage{fancyhdr} % For header and footer
% % tocloft を include すると目次の番号と文章が被る?
%\usepackage{tocloft} % 図目次、表目次の見た目を変えるため
\usepackage{pxrubrica} % For writing ruby

% % % Convenient for document of science and technology
\usepackage{numprint} % 数値の整形
\usepackage{siunitx} % For using SI units
\usepackage{listings} % For inserting programming code

% % % Option
\usepackage{docmute} % For compilation of divided files
\usepackage{bxcoloremoji} % For using emoji

% % % Reference for packages
% % empheq.sty
% http://muscle-keisuke.hatenablog.com/entry/2015/11/23/122725
% % geometry.styについて
% http://joker.hatenablog.com/entry/2012/07/09/153537
% % tocloft.styについて
% http://www.biwako.shiga-u.ac.jp/sensei/kumazawa/tex/tocloft.html
% http://tex.stackexchange.com/questions/20337/adding-word-table-before-each-entry-in-list-of-tables
% % makeidx.sty
% http://www.biwako.shiga-u.ac.jp/sensei/kumazawa/tex/makeidx.html
% % pxrubrica.sty
% https://qiita.com/zr_tex8r/items/42466cbcbeb670a3a2dc
% % %

% % % 横にenumerateを並べるため
\usepackage{tablists}

% % % AMS-Theorem Environment
\theoremstyle{definition}

% 試験的に導入(定義を枠で囲むため).2005年だからパッケージとしては古いか?(もっといいのがあるかも)
% amsthm環境と競合するため,設定したあとで読み込む.
% https://ctan.org/tex-archive/macros/latex/contrib/thmbox
%\usepackage[nocut, nounderline]{thmbox}
%\thmboxoptions{titlestyle={\,(\textbf{#1})}}
%\thmboxoptions{bodystyle=\upshape\noindent}
%\newtheorem[S]{question}{問題}

% % % Config for hyperref
\hypersetup{ %
	breaklinks=true, %
	colorlinks=false, %
	urlcolor=blue, %
	urlbordercolor={0 1 1}}

% % % Define Command
\newcommand{\bm}[1]{\mbox{\boldmath $#1$}}
\newcommand{\dif}{\mathrm{d}}
\newcommand{\set}[1]{\left\{\,#1\,\right\}}
\newcommand{\im}{\operatorname{Im}}
\newcommand{\dom}{\operatorname{dom}}
\newcommand{\cod}{\operatorname{cod}}

\newcommand{\mySpace}{\phantom{\rule[-1cm]{2cm}{2cm}}}

% % % Re-define names

\begin{document}
	% % % Begin of Body
	% Chapterごとにファイルを分けて,それぞれをincludeする
	%\title{数学II 中間試験}
\date{}
\maketitle

\vspace*{-2cm}
\begin{table}[!h]
	\begin{flushright}
		\begin{tabular}{rc}
			\vspace*{0.1in} & \usdate\today\\
			学籍番号 & \vspace{0.1in} \\
			名前 & \vspace{0.1in} \\
		\end{tabular}
	\end{flushright}
\end{table}

\begin{itembox}[c]{解答上の注意}
	\begin{itemize}
		\item 設問に指定がない場合は,計算過程も含めて,記述してください.
		\item 穴埋め問題は,解答だけを,記入してください.
		\item 試験時間は,90分です.
		\item 大問7と大問8は,選択問題です.どちらか1問を選んで解答してください.(両方解答しても構いません.得点を上乗せします.)
	\end{itemize}
\end{itembox}

\newpage
\begin{questions}
	\question
	次の問いに答えなさい.
	\begin{parts}
		\part
		以下は,三角関数の値を表にしたものである.空欄を埋めよ.
		\begin{table}[!h]
			%\centering
			\begin{tabular}{c|c|c|c|c|c}
				$\theta$ & $0$ & $\pi/6$ & $\pi/4$ & $\pi/3$ & $\pi/2$ \\
				\hline
				$\sin\theta$ & \mySpace & & & \mySpace & \\
				\hline
				$\cos\theta$ & & \mySpace & & & \mySpace \\
				\hline
				$\tan\theta$ & & & \mySpace & & \\
			\end{tabular}
		\end{table}
		\begin{table}[!h]
			%\centering
			\begin{tabular}{c|c|c|c|c}
				$\theta$ & $2\pi/3$ & $3\pi/4$ & $5\pi/6$ & $\pi$ \\
				\hline
				$\sin\theta$ & \mySpace & & & \mySpace \\
				\hline
				$\cos\theta$ & & \mySpace & & \\
				\hline
				$\tan\theta$ & & & \mySpace & \\
			\end{tabular}
		\end{table}
	
		\part
		以下の値を求めよ.
		\begin{subparts}
			\subpart $\displaystyle \arcsin\frac{1}{2}$
			\answerline
			\subpart $\displaystyle \arccos\left(-\frac{1}{\sqrt{2}}\right)$
			\answerline
			\subpart $\displaystyle \arctan\sqrt{3}$
			\answerline
		\end{subparts}
	\end{parts}

	\newpage
	\question
	次の問いに答えなさい.
	\begin{parts}
		\part
		定義に従って$f(x) = 1/x$の微分を求めなさい.
		\makeemptybox{7cm}
		
		\part
		$f'(2)$の値を求めなさい
		\answerline
		
		\part
		点$x = 2$における接線の方程式を求めなさい.
		\makeemptybox{7cm}
	\end{parts}

	\newpage
	\question
	関数$f(x) = |x|$は,点$x = 0$において微分不可能である.その理由を記述しなさい.
	\makeemptybox{18cm}
	
	\newpage
	\question
	次の問いに答えなさい.
	\begin{parts}
		\part
		三角関数の加法定理を記述しなさい.
		\begin{subparts}
			\subpart
			$\sin\left(\alpha+\beta\right)$ = \mySpace
			\subpart
			$\cos\left(\alpha+\beta\right)$ = \mySpace	
		\end{subparts}
	
		\part
		三角関数の加法定理を用いて,以下の和積公式を導出しなさい.
		\[
			\sin A - \sin B = 2\cos\left(\frac{A+B}{2}\right)\sin\left(\frac{A-B}{2}\right)
		\]
		\makeemptybox{14cm}
		
		\part
		定義に従って$f(x) = \sin x$の微分を求めなさい.必要に応じて$\displaystyle \lim\limits_{x \to 0}\frac{\sin x}{x} = 1$を用いてもよい.
		\makeemptybox{18cm}
	\end{parts}

	\newpage
	\question
	%逆関数の微分を用いて,$f(x) = a^x$の微分を求めなさい.
	%\makeemptybox{18cm}
	\setlength\answerlinelength{3in}
	次の関数をそれぞれ微分しなさい.
	\begin{parts}
		\part
		$\displaystyle e^{(2-3x)}$
		\answerline
		
		\part
		$\displaystyle (x^2-x)\ln x$
		\answerline
		
		\part
		$\displaystyle \log_{2}x$
		\answerline
		
		\part
		$\displaystyle x\cos x - \sin x$
		\answerline
	\end{parts}
	\setlength\answerlinelength{1in}
	
	\newpage
	\question
	次の問いに答えなさい.
	\begin{parts}
		\part
		ライプニッツの公式を用いて,$f(x) = x^22^x$の$n$階導関数を求めなさい.
		\makeemptybox{14.5cm}
		
		\part
		$f(x)$の1階導関数$f^{(1)}$,2階導関数$f^{(2)}$を求めなさい.
		\makeemptybox{5cm}
		
		%\part
		%関数の増減表を作り,グラフの概形を描きなさい.
		%\makeemptybox{10cm}
	\end{parts}

	\newpage
	\question
	関数$\displaystyle f(x) = e^{-x^2}$の極値を調べ,グラフの概形を描きなさい.
	
	\newpage
	\question
	関数$\displaystyle f(x) = x^4-6x^2-8x+6$の極値を調べ,グラフの概形を描きなさい.また,閉区間$[0, 3]$における最大値・最小値を求めなさい.
\end{questions}
	\title{数学II 期末試験}
\date{}
\maketitle

\vspace*{-2cm}
\begin{table}[!h]
	\begin{flushright}
		\begin{tabular}{rc}
			\vspace*{0.1in} & \usdate\today\\
			学籍番号 & \vspace{0.1in} \\
			名前 & \vspace{0.1in} \\
		\end{tabular}
	\end{flushright}
\end{table}

\begin{itembox}[c]{解答上の注意}
	\begin{itemize}
		\item 設問に指定がない場合は,計算過程も含めて,記述してください.
		\item 穴埋め問題は,解答だけを,記入してください.また,余白に計算過程が記述されている場合,部分点を考慮する可能性があります.
		\item 試験時間は,90分です.
		%\item 大問7と大問8は,選択問題です.どちらか1問を選んで解答してください.(両方解答しても構いません.得点を上乗せします.)
	\end{itemize}
\end{itembox}

\newpage
\begin{questions}
	\question
	以下は,関数$f(x)$について,その導関数$f'(x)$と不定積分$\int f(x)\,\dif x$を表にしたものである.空欄を埋めよ.(ただし,積分定数は省略してよい.)
	\begin{table}[!h]
		\centering
		\begin{tabular}{c|c|c}
			$f'(x)$ & $f(x)$ & $\int f(x)\,\dif x$ \\
			\hline
			\mySpace & $x^a\quad(a \neq -1)$ & \\
			\hline
			& $\displaystyle \frac{1}{x}$ & \mySpace \\
			\hline
			\mySpace & $\sin x$ & \\
			\hline
			& $\cos x$ & \mySpace \\
			\hline
			\mySpace & $\tan x$ & $-\ln\left|\cos x\right|$ \\
			\hline
			& $e^x$ & \mySpace \\
			\hline
			\mySpace & $\ln x$ & \\
		\end{tabular}
	\end{table}
	
	\newpage
	\question
	\setlength\answerlinelength{3in}
	以下の関数をそれぞれ微分しなさい.
	\begin{parts}
		\part
		$\displaystyle x^3 + 3x^2 -\frac{2}{x}$
		\answerline
		
		\part
		$\displaystyle \sqrt{1 - x^2}$
		\answerline
		
		\part
		$\displaystyle e^{2x - 1}$
		\answerline
		
		\part
		$\displaystyle x\sin x - \cos x$
		\answerline
		
		%\part
		%$\displaystyle \frac{1}{1 + \cos x}$
		%\answerline
		
		\part
		$\displaystyle \left(\ln x\right)^2$
		\answerline
	\end{parts}

	\newpage
	\question
	以下の曲線$y = f(x)$について,指示された値に対応する点における接線の方程式を求めよ.
	\begin{parts}
		\part
		$\displaystyle f(x) = 3x^2\quad(x = 2)$
		\answerline
		
		\part
		$\displaystyle f(x) = \sqrt{x}\quad(x = 4)$
		\answerline
		
		%\part
		%$\displaystyle f(x) = \ln(2x - 1)\quad(x = 1)$
		%\answerline
		
		\part
		$\displaystyle f(x) = xe^x\quad(x=0)$
		\answerline
	\end{parts}
	\setlength\answerlinelength{1in}
	
	\newpage
	\question
	関数$\displaystyle f(x) = x^4 + 4x^3 - 16x - 3$の極値を調べ,グラフの概形を描きなさい.
	
	\newpage
	\question
	\setlength\answerlinelength{3in}
	以下の関数をそれぞれ不定積分しなさい.
	\begin{parts}
		\part
		$\displaystyle x^3 + 3x^2 -\frac{2}{x}$
		\answerline
				
		\part
		$\displaystyle \frac{3}{2\sqrt{x}}$
		\answerline

		\part
		$\displaystyle 2e^{2x} + 3^x$
		\answerline
		
		\part
		$\displaystyle 3\sin 3x - \cos(x - 2)$
		\answerline
	\end{parts}

	\newpage
	\question
	以下の関数をそれぞれ不定積分しなさい.
	\begin{parts}
		%\part
		%$\displaystyle x(1 - x)^5$
		%\answerline

		\part
		$\displaystyle x\sqrt{x + 1}$
		\answerline

		\part
		$\displaystyle x\cos x$
		\answerline
		
		\part
		$\displaystyle xe^x$
		\answerline
	\end{parts}
	
	\newpage
	\question
	以下の定積分の値をそれぞれ求めなさい.
	\begin{parts}
		\part
		$\displaystyle \int_1^2 \left(2x + \frac{1}{x}\right)\dif x$
		\answerline
		
		\part
		$\displaystyle \int_0^2 x(2 - x)^3\,\dif x$
		\answerline
		
		\part
		$\displaystyle \int_0^{\frac{\pi}{2}}\sin^2x\cos x\,\dif x$
		\answerline
		
		\part
		$\displaystyle \int_1^3 x\ln x\,\dif x$
		\answerline
	\end{parts}
	\setlength\answerlinelength{1in}

	\newpage
	\question
	関数$\displaystyle f(x) = x^3 + x^2 - 4x$について,点$x = 1$における接線の方程式を求めなさい.また,その接線と曲線$y = f(x)$とで囲まれた面積を求めなさい.
	
	\newpage
	\question
	関数$\displaystyle f(x) = x^3 - x^2 - x + 1$について,方程式$f(x) = 0$の解を求めなさい.また,曲線$y = f(x)$と$x$軸とで囲まれた面積を求めなさい.
\end{questions}
	% % % End of Body
\end{document}
