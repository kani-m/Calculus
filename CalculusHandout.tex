% % % Document Type
% http://www.biwako.shiga-u.ac.jp/sensei/kumazawa/tex/book.html
% http://ichiro-maruta.blogspot.jp/2013/03/latex.html
\RequirePackage[l2tabu, orthodox]{nag}
\documentclass[12pt, a4j, openany, uplatex, dvipdfmx]{jsbook}

% % % Paper size
%\usepackage[top=1truein, bottom=1truein, left=1.5truein, right=1truein]{geometry} % 用紙サイズの設定のため

% % % Necessary
\usepackage{amssymb, amsmath, amsthm, latexsym} % For mathematics [Before hyperref]
\usepackage[dvipdfmx,setpagesize=false]{hyperref} % For inserting hyperlink [Before graphicx]
\usepackage{pxjahyper} % https://texwiki.texjp.org/?hyperref
\usepackage[dvipdfmx]{graphicx} % For inserting figure
\usepackage[labelformat=simple]{subcaption} % For caption of figures and tables
%\renewcommand\thesubfigure{(\alph{subfigure})}
%\renewcommand\thesubtable{(\alph{subtable})}
\usepackage[usenames,dvipdfmx]{color} % For coloring
\usepackage[shortlabels]{enumitem} % For useful enumerate and itemize environment
\usepackage{comment} % For commenting multiline
\usepackage{cite} % For citing references [After hyperref]
\usepackage{url} % For inserting URL
\usepackage{acro} % For listing abbreviations and symbols
\usepackage{makeidx} % For making index

% % % Convenient for graph
%\usepackage{tikz-cd} %load package after color.sty & soul.sty

% % % Convenient for tables and emphasis
\usepackage{tabularx} % For flexible table
\usepackage{multirow} % For combining multi rows for table
%\usepackage{colortbl} % For coloring table
\usepackage{longtable} % For long table across multiple pages
%\usepackage[normalem]{ulem} % For flexible underline etc.
%\usepackage{ascmac} % For framing multiline -> bxascmac に変更したほうがよい?

% % % 文書の見た目のためのパッケージ
\usepackage{setspace} % For single space or double space
\usepackage{afterpage} % For inseting newpage 
\usepackage{datetime} % For date style of title
\usepackage{fancyhdr} % For header and footer
% % tocloft を include すると目次の番号と文章が被る?
%\usepackage{tocloft} % 図目次、表目次の見た目を変えるため
\usepackage{pxrubrica} % For writing ruby

% % % Convenient for document of science and technology
\usepackage{numprint} % 数値の整形
\usepackage{siunitx} % For using SI units
\usepackage{listings} % For inserting programming code

% % % Option
\usepackage{docmute} % For compilation of divided files
\usepackage{bxcoloremoji} % For using emoji

% % % Reference for packages
% % geometry.styについて
% http://joker.hatenablog.com/entry/2012/07/09/153537
% % tocloft.styについて
% http://www.biwako.shiga-u.ac.jp/sensei/kumazawa/tex/tocloft.html
% http://tex.stackexchange.com/questions/20337/adding-word-table-before-each-entry-in-list-of-tables
% % makeidx.sty
% http://www.biwako.shiga-u.ac.jp/sensei/kumazawa/tex/makeidx.html
% % pxrubrica.sty
% https://qiita.com/zr_tex8r/items/42466cbcbeb670a3a2dc
% % %

% % % AMS-Theorem Environment
\theoremstyle{definition}
\newtheorem{theorem}{定理}[chapter]
\newtheorem*{theorem*}{定理}
\newtheorem{prop}[theorem]{命題}
\newtheorem{lemma}[theorem]{補題}
\newtheorem{cor}[theorem]{系}
\newtheorem{example}{例}[theorem]
\newtheorem*{example*}{例}
\newtheorem*{definition*}{定義}
\newtheorem*{rem*}{注意}
\newtheorem{guide}[theorem]{参考}

% 試験的に導入(定義を枠で囲むため).2005年だからパッケージとしては古いか?(もっといいのがあるかも)
% amsthm環境と競合するため,設定したあとで読み込む.
% https://ctan.org/tex-archive/macros/latex/contrib/thmbox
\usepackage[nocut, nounderline]{thmbox}
\thmboxoptions{titlestyle={\,(\textbf{#1})}}
\newtheorem[M]{definition}[theorem]{定義}

% % % Config for hyperref
\hypersetup{ %
	breaklinks=true, %
	colorlinks=false, %
	urlcolor=blue, %
	urlbordercolor={0 1 1}}

% % % Config for header of chapter
% % http://oversleptabit.com/?p=390
\makeatletter
\def\@makechapterhead#1{%
	\vbox to 40mm{\kern7mm\parindent\z@
		\raggedright
		\reset@font
		\huge\bfseries
		\ifnum \c@secnumdepth >\m@ne
		\hskip10mm{\fontsize{20Q}{20Q}\selectfont\@chapapp
			{\Huge\thechapter}\@chappos}
		\vskip1.5mm
		\hrule \@height 0.1mm
		\vskip.5mm
		\hrule \@height 0.5mm
		\vskip2mm
		\begin{flushright}
			\fontsize{28Q}{29Q}\bfseries\selectfont#1\par
		\end{flushright}
		\vfill
		\fi}
	\nobreak
}
\makeatother

% % % Define Command
\newcommand{\bm}[1]{\mbox{\boldmath $#1$}}
\newcommand{\dif}{\mathrm{d}}
\newcommand{\set}[1]{\left\{\,#1\,\right\}}
\newcommand{\im}{\operatorname{Im}}
\newcommand{\dom}{\operatorname{dom}}
\newcommand{\cod}{\operatorname{cod}}

% % % Re-define names

% % % Reference: 
% http://tex.stackexchange.com/questions/86666/how-to-create-both-list-of-abbreviations-and-list-of-nomenclature-using-nomencl
% ftp://ftp.kddilabs.jp/CTAN/macros/latex/contrib/acro/acro_en.pdf
% % %

% Configuration for hyperlink and list of acronyms
\acsetup{hyperref=true, list-style=longtable, list-heading=chapter*}

% class `abbrev': abbreviations:
\DeclareAcronym{vs}{
  short = VS ,
  long  = Vector Space ,
  class = abbrev
}


% class `nomencl': nomenclature (symbol)
\DeclareAcronym{M}{
  short = $\bm{M}$ ,
  long  = 行列 ,
  sort  = A ,
  class = nomencl
}
\DeclareAcronym{v}{
  short = $\bm{v}$ ,
  long  = ベクトル ,
  sort  = A ,
  class = nomencl
}
\DeclareAcronym{R}{
  short = $\mathbb{R}$ ,
  long  = 実数 ,
  sort  = R ,
  class = nomencl
}

 % List of abbreviations and symbols
\makeindex
\begin{document}
	% % % Begin of Front matter
	% 表題、目次、序文等からなるフロントマター部用の設定をする
	\frontmatter
	% 表紙
	\title{微積分}
	\author{Yuki M.}
	\date{\usdate\today}
	\maketitle
	% 目次
	%\setcounter{tocdepth}{3}
	\tableofcontents
	% 図、表、略語、記号一覧
	%\listoffigures
	%\listoftables
	%\printacronyms[include-classes=abbrev,name=List of Abbreviations]
	%\printacronyms[include-classes=nomencl,name=List of Symbols]
	% 略語一覧も記号一覧も表示しない場合,参照を切るために以下を設定する
	\acsetup{hyperref=false}
	% % % End of Front matter

	% % % Begin of Body
	\mainmatter
	% Chapterごとにファイルを分けて,それぞれをincludeする
	\chapter{はじめに}
この資料は,微積分学の講義のための参考資料です.自分の備忘録も兼ねているのでやや冗長な表現もあるやもしれません.
筆者もまだまだ勉強中であるため,理解が追いつかず根本的に誤解している箇所があるかと思います.
ぜひとも,「ここの文章はおかしいのではないか?」等のご指摘をお待ちしております.
また,誤字・脱字,および論理的な誤りがあればご一報ください.

\section{微積分とは?}
微積分は線形代数と並び現代数学の基礎として,大学など高等教育機関において広く学ばれている教養科目である.
また,線形代数と同様に,自然科学・工学をはじめとした応用科学・社会科学など幅広い分野で応用されている計算ツールである.

そもそも微積分とは,簡単に言ってしまえば「変化」を捉えるための数学である.世の中の物理量(例えば,物体の速度や温度など)のほとんどは,位置や時間の変化に伴って変化する.その変化の度合いが,微かな変化の中にどれぐらいであったのかを,捉える数学が「微分」であり,変化の度合いの積み重ねから異なる位置・時間における物理量を予測する数学が「積分」である.

% % % 歴史的背景や具体的な応用例など,導入部を厚くしたい
% 歴史的には,17世紀にニュートンが作り出したと言われており,微積分の概念は,その後の数学に大きな影響を与えたとされている.

\section{講義の内容}
この講義においては,まず始めに表\ref{table:ele_func}に示すような初等関数を学ぶ.
これは「変化」を捉える数学である「微積分」を学ぶ上で,大きな寄り道に思えるかもしれない.実際のところ「微分」「積分」の概念を理解することと,初等関数を理解することは,全く別の話である.しかし,これらの初等関数は,様々な現象を数式にモデル化する上で使用されるものであり,実用上において微積分と初等関数は切っても切れない関係にあることを理解してほしいため,まず始めに取り上げるものとする.

\begin{table}[!h]
	\centering
	\caption{初等関数}
	\label{table:ele_func}
	\begin{tabular}{c|c}
		& 例 \\
		\hline
		代数関数 & $x^2 + x - 2$ \\
		指数関数 & $2^x$ \\
		対数関数 & $\log_2{x}$ \\
		三角関数 & $\sin x, \cos x, \tan x$ \\
		逆三角関数 & $\arcsin x, \arccos x, \arctan x$ \\
	\end{tabular}
\end{table}

次に,微積分を学ぶ上で核となる「変化」を明確に記述するために極限と呼ばれる概念を取り上げる.つまり,微分とは「微かな変化」を捉えるための数学のことであったが,「``微かな''変化」とは一体どのようなものなのか,微小量と呼ばれる考え方を導入し,微分が持つ本質的な意味を探っていく.

その後,定積分と呼ばれる微小量の積み重ねによって表される計量が,微分の逆演算としての不定積分を用いて計算した結果と等しくなる微分積分学の基本定理について取り上げ,積分が表す2通りの意味を理解し,先人たちが築き上げた大きな金字塔を追体験していく.

またこの講義では,微分・積分の概念の意味を理解すると同時に,初等関数の基本的な微分・積分の計算を行っていく.
具体的には,様々な初等関数のグラフの概形や極値,様々な図形の面積・体積を求めるといった問題を想定している.

その上で,実用上欠かすことができない微分方程式や級数について触れ,もし時間があれば,多変数の微分・積分の初歩的概念について取り上げることとする.

\section{到達目標}
この講義では,以下に示す項目を達成することを目的とする.
\begin{enumerate}[(1)]
	\item 微積分が持つ本質的な意味をつかむ.
	\item 微積分に関する基本的な計算ができるようになる.
	\item 微積分の知識を使い,初等関数のグラフの概形や基本的な図形の計量が求められるようになる.
	\item 単純な微分方程式が解けるようになる.
	\item[(*5)] 各定理の主張が理解でき,その証明が追えるようになる.
\end{enumerate}

(1)(2)については,受講者全員に達成してほしい目標として設定した.(3)(4)については,講義全体を通して身につけて欲しい計算力に関する目標である.(*5)については,高度な目的となるため,深くは言及しないが,論理的な思考力を身につけるため,ぜひチャレンジして欲しい目標である.
%微積分は,統計学をはじめとした様々な学問分野で欠かせない計算ツールであるため,毎回の演習を通して理解の深化と計算力の向上を促したいと考えている.

\section{評価基準}
表\ref{table:criteria}にこの講義の評価基準を示す.この内,筆記試験については,中間試験を実施するかどうか履修人数が確定したタイミングで話し合いたいと考えている.また,実施する場合は,中間試験と期末試験の割合の比率についても同様に話し合いたい.

\begin{table}[!h]
	\centering
	\caption{評価項目とその割合}
	\label{table:criteria}
	\begin{tabular}{c|c}
		評価項目 & 割合 \\
		\hline
		講義への出席 & $\SI{30}{\percent}$ \\
		毎回の課題 & $\SI{30}{\percent}$ \\
		筆記試験 & $\SI{40}{\percent}$ \\
	\end{tabular}
\end{table}

\section{日程と進め方}
表\ref{table:schedule}にこの講義の日程を示す.原則として毎週水曜日と金曜日の6限(17:40 -- 19:10)に講義を行う.進め方としては,1コマをさらに2つに分割して,45分間を1つのブロックとし,新しい概念の説明と,それに付随する演習問題の解説を当てる予定である.
%高等学校における講義と演習が時間的に密接にある授業を目指していきたいと考えている.
毎回の課題については,小テストにするか宿題形式にするか現時点(2017/9/28)では決定していない.宿題形式の場合は,別途Webサービスを通して提出という形を取っていきたい.

また,仮に中間試験を行うとすれば,11月下旬を想定している.

\begin{table}[!h]
	\centering
	\caption{講義の日程とテーマ}
	\label{table:schedule}
	\begin{tabular}{c|c}
		日付 & 内容 \\
		\hline
		10/04 & \multirow{4}{*}{初等関数の復習} \\
		10/06 & \\
		10/11 & \\
		10/13 & \\
		\hline
		10/18 & \multirow{2}{*}{極限と導関数} \\
		10/20 & \\
		\hline
		10/25 & \multirow{4}{*}{初等関数の微分} \\
		10/27 & \\
		11/01 & \\
		11/08 & \\
		\hline
		11/10 & \multirow{2}{*}{高階導関数} \\
		11/15 & \\
		\hline
		11/17 & \multirow{2}{*}{初等関数のグラフの概形と極大値・極小値} \\
		11/24 & \\
		\hline
		11/29 & \multirow{2}{*}{定積分と原始関数} \\
		12/01 & \\
		\hline
		12/06 & \multirow{4}{*}{初等関数の積分} \\
		12/08 & \\
		12/13 & \\
		12/15 & \\
		\hline
		12/20 & \multirow{2}{*}{図形の面積・体積と広義積分} \\
		12/22 & \\
		\hline
		12/27 & \multirow{2}{*}{微分方程式} \\
		01/05 & \\
		\hline
		01/10 & \multirow{2}{*}{級数} \\
		01/12 & \\
		\hline
		01/17 & \multirow{2}{*}{偏微分} \\
		01/19 & \\
		\hline
		01/24 & \multirow{2}{*}{重積分} \\
		01/26 & \\
	\end{tabular}
\end{table}

	\chapter{初等関数}
この章では,中学校・高等学校で学んだ\textbf{関数}について改めて定義を確かめ,その例である初等関数を個別に紹介する.

\section{関数とは}
数学における関数とは,ある集合から別の集合への対応関係を示したものである.イメージとして図\ref{fig:blackbox}のようなブラックボックス(中身が分からない箱)を思い浮かべるとよい.ここでは,改めて用語を定義し,数学における関数を形作っていきたい.

\vspace*{4cm}
\begin{figure}[!h]
	\caption{関数のイメージ}
	\label{fig:blackbox}
\end{figure}

\begin{definition}[集合] % set
	\textbf{集合}とは,ある対象(それは数であっても関数であっても,あるいは果物などであってもよい)を集めたものである.集合を構成する1つ1つの対象を\textbf{元}あるいは\textbf{要素}と呼ぶ.このとき,要素の順序や重複は気にしないものとする.また,対象が集合の要素であるかどうか真偽が決定できるように定義されなければならない.とくに,対象$a$が集合$A$の要素であるとき$a \in A$と表記し,要素でないとき$a \notin A$と表記する.
\end{definition}
\begin{rem*}
	$a \in A$の読み方としては,「$a$は$A$の要素である」以外にも「$a$は$A$に属している」「$A$は$a$を要素として持つ」などがある.
\end{rem*}
\begin{example*}
	$\set{1,2,3}$や$\set{\text{りんご}, \text{ばなな}, \text{みかん}}$は,集合の例である.このように具体的な要素を列挙することを,\ruby{{\textbf{外}}{\textbf{延}}}{がい|えん}記法という.また,$\set{x \mid x \text{は10以下の正の偶数}}$も集合の例である.このように集合に属する要素が満たすべき条件を明示することを,\ruby{{\textbf{内}}{\textbf{包}}}{ない|ほう}記法という.
\end{example*}
% % % 太字は,読者の目を止まらせるためのもの->ふりがなが重要でない場合には太字にしなくてもよい?
\begin{example*}
	$\set{1,1,3}$と$\set{1,3,1}$と$\set{1,3}$は,同一の集合である.そのため,通常は,最も少ない要素数になるように,特定の順序に従って表記される.
\end{example*}
\begin{example*}
	数の集合について,一般的に使われている記号を表\ref{table:number}に示す.これら数の集合のように,要素数が無限になる集合もあり,ほとんどの場合,内包記法によって表記される.
	\begin{table}[!h]
		\centering
		\caption{数の集合に用いられる記号}
		\label{table:number}
		\begin{tabular}{cc}
			$\mathbb{N}$ & 自然数全体の集合 \\
			$\mathbb{Z}$ & 整数全体の集合 \\
			$\mathbb{Q}$ & 有理数全体の集合 \\
			$\mathbb{R}$ & 実数全体の集合 \\
			$\mathbb{C}$ & 複素数全体の集合 \\
		\end{tabular}
	\end{table}
\end{example*}
\begin{definition}[空集合] % empty set
	集合において,要素が1つもない空っぽな集合を考えることができる.これを\textbf{空集合}といい,$\emptyset$と表記する.
\end{definition}
% % % 閉区間・開区間を,集合の例として紹介する?
% % % 開閉区間を取り入れるなら,上限・下限,最大・最小も取り入れたい.
% % % ->代数関数の項に移動

\begin{definition}[部分集合] % subset
	集合$X$が,集合$S$の\textbf{部分集合}であるとは,$X$の要素が全て$S$に属することである.このとき$X \subseteq S$と表記する.
\end{definition}
\begin{rem*}
	$X \subseteq S$の読み方としては,「$X$は$S$の部分集合である」以外にも「$X$は$S$に含まれる」「$S$は$X$を包含する」などがある.また,$X$が$S$の部分集合でない場合には,$X \nsubseteq S$と表記される.
\end{rem*}
\begin{example*}
	$\set{1,2,3}$は,自然数全体の集合$\mathbb{N}$の部分集合である.また,$\mathbb{N}$は,整数全体の集合$\mathbb{Z}$の部分集合である.
\end{example*}
\begin{example*}
	集合$S$について,空集合$\emptyset$と自分自身$S$は,常に部分集合である.
\end{example*}

複数の集合を用いて,新しい集合を作ることができる.ここでは,代表的なものを定義する.
\begin{definition}[和集合] % union, join
	\textbf{和集合}$A \cup B$とは,集合$A$と集合$B$に属する要素を集めた集合のことである.
\end{definition}
\begin{definition}[共通部分] % intersection
	\textbf{共通部分}$A \cap B$とは,集合$A$と集合$B$に\underline{同時に}属する要素を集めた集合のことである.
\end{definition}
\begin{rem*}
	$\cup$は「または」と読み,$\cap$は「かつ」と読む.
\end{rem*}
\begin{example*}
	集合$A=\set{1,2,3,4,5}$と集合$B=\set{3,4,5,6,7}$に対し,和集合$A \cup B$は$\set{1,2,3,4,5,6,7}$であり,共通部分$A \cap B$は$\set{3,4,5}$である.
\end{example*}

\begin{definition}[差集合] % set difference
	\textbf{差集合}$A \setminus B$とは,集合$A$に属する要素のうち,集合$B$に属していない要素を集めた集合のことである.
\end{definition}
\begin{example*}
	集合$A=\set{1,2,3,4,5}$と集合$B=\set{3,4,5,6,7}$に対し,差集合$A \setminus B$は$\set{1,2}$である.整数全体の集合$\mathbb{Z}$と集合$\set{0}$に対し,差集合$\mathbb{Z} \setminus \set{0}$は$\set{x \mid \text{$x$は正の整数または負の整数}}$である.
\end{example*}

ここまで定義した集合について,図示したものが図\ref{fig:VennDiagram}である.

\vfill
\begin{figure}[!h]
	\caption{集合の図示}
	\label{fig:VennDiagram}
\end{figure}
\clearpage

\begin{definition}[順序組] % ordered tuple, ordered list, ordered pair
	\textbf{順序組}とは,ある対象を集めたものである.このとき,集合とは違い,要素の順序や重複に意味があるものとする.
\end{definition}
\begin{example*}
	$(1, 2)$や$(\spadesuit, \text{K})$は,順序組の例である.このような2つの要素から成る順序組を\textbf{順序対}と呼ぶ.また,$(1, 2)$と$(2, 1)$は,異なる順序組であり,$(1, 1, 3)$と$(1, 3)$も異なる順序組である.
\end{example*}
\begin{definition}[直積集合] % (direct) product
	\textbf{直積集合}$A \times B$とは,集合$A$に属する要素と,集合$B$に属する要素から,新たに順序対を作り,それら全てを要素とする集合のことである.このとき,$A \times B$の要素である順序対$(a, b)$は,直積集合の表記通りの順序を保たなければならない.つまり,$a \in A, b \in B$である.
\end{definition}
\begin{example*}
	集合$A=\set{1,2,3}$と集合$B=\set{\heartsuit,\diamondsuit}$に対し,直積集合$A \times B$は$\set{(1, \heartsuit),(1,\diamondsuit),(2, \heartsuit),(2,\diamondsuit),(3, \heartsuit),(3,\diamondsuit)}$である.$(\heartsuit, 1)$など順序を逆にしたものは,$A \times B$の要素ではない.
\end{example*}
% % % 全体集合,補集合,対称差,冪集合,商集合は,今回の微積分に出てこないかも?のでカット

この直積集合を用いて,2つの集合間に二項関係を築くことができる.
\begin{definition}[二項関係] % binary relation
	集合$A, B$間の\textbf{二項関係}$R$とは,直積集合$A \times B$の部分集合である.このとき,$R$の要素である順序対$(a,b)$を,$aRb$と表記することもある.また,集合$A$を\textbf{始集合}と呼び,集合$B$を\textbf{終集合}と呼ぶ.
\end{definition}
\begin{example*}
	集合$A=\set{1,2,3,4,5}$と集合$B=\set{3,4,5,6,7}$に対し,$aRb$を$a$は$b$を割り切る関係とする.このとき,関係$R$は,
	\[
		\set{(1,3),(1,4),(1,5),(1,6),(1,7),(2,4),(2,6),(3,3),(3,6),(4,4),(5,5)}
	\]
	である.
\end{example*}
% % % 反射的,対称的,推移的,同値関係
% % % 単射,全射,全単射
% % % 二項関係は,二項演算を含み,二項演算と二項算法はニアイコールである.

% 通常,二項関係は,様々な例を作ることができる.ここに一定の制約を加えることにより,写像を定義することができるようになる.
また,集合$A, B$間の二項関係$R$について,一定の条件を満たすことで,性質が定義される.
\begin{definition}[一意性]
	\label{def:uniqueness}
	\begin{align*}
	\text{$R$は単射(左一意的)である.} &\Leftrightarrow \text{集合$B$の要素$b$に対し,$aRb$が存在するならば,} \\
	&\qquad \text{$a$は一意的に決定できる.} \\
	\text{$R$は関数的(右一意的)である.} &\Leftrightarrow \text{集合$A$の要素$a$に対し,$aRb$が存在するならば,} \\
	&\qquad \text{$b$は一意的に決定できる.} \\
	\text{$R$は一対一である.} &\Leftrightarrow \text{$R$は左一意的かつ右一意的である.}
	\end{align*}
\end{definition}
\begin{definition}[全域性]
	\label{def:total}
	\begin{align*}
	\text{$R$は全域的(左全域的)である.} &\Leftrightarrow \text{集合$A$の要素$a$に対し,集合$B$の要素$b$が存在して,} \\
	&\qquad \text{二項関係$R$は$aRb$を要素として持つ.} \\
	\text{$R$は全射(右全域的)である.} &\Leftrightarrow \text{集合$B$の要素$b$に対し,集合$A$の要素$a$が存在して,} \\
	&\qquad \text{二項関係$R$は$aRb$を要素として持つ.} \\
	\text{$R$は対応である.} &\Leftrightarrow \text{$R$は左全域的かつ右全域的である.}
	\end{align*}
\end{definition}
\begin{definition}[写像および全単射]
	\label{def:specialRelations}
	\begin{align*}
	\text{$R$は写像(一意対応)である.} &\Leftrightarrow \text{$R$は関数的(右一意的)かつ全域的(左全域的)である.} \\
	\text{$R$は全単射(双射)である.} &\Leftrightarrow \text{$R$は単射かつ全射である.}
	\end{align*}
\end{definition}
\begin{comment}
\begin{definition}[一意性]
	\begin{align*}
		\text{$R$は単射(左一意的)である.} &\Leftrightarrow \forall a_1, a_2 \in A, \forall b \in B \left[a_1Rb \wedge a_2Rb \Rightarrow a_1 = a_2 \right] \\
		\text{$R$は関数的(右一意的)である.} &\Leftrightarrow \forall a \in A, \forall b_1, b_2 \in B \left[aRb_1 \wedge aRb_2 \Rightarrow b_1 = b_2 \right] \\
		\text{$R$は一対一である.} &\Leftrightarrow \text{$R$は左一意的かつ右一意的である.}
	\end{align*}
\end{definition}
\begin{definition}[全域性]
	\begin{align*}
	\text{$R$は全域射(左全域的)である.} &\Leftrightarrow \forall a \in A, \exists b \in B \text{ s.t. } aRb \\
	\text{$R$は全射(右全域的)である.} &\Leftrightarrow \forall b \in B, \exists a \in A \text{ s.t. } aRb \\
	\text{$R$は対応である.} &\Leftrightarrow \text{$R$は左全域的かつ右全域的である.}
	\end{align*}
\end{definition}
\end{comment}
定義\ref{def:uniqueness}と定義\ref{def:total}の一部を図示したものが,図\ref{fig:relationProperty}である.

\vfill
\begin{figure}[!h]
	\caption{定義\ref{def:uniqueness}と定義\ref{def:total}の図示}
	\label{fig:relationProperty}
\end{figure}
\afterpage{\clearpage}
\newpage

定義\ref{def:specialRelations}における写像の定義を言い換えたものが,定義\ref{def:mapping}である.
\begin{definition}[写像] % mapping, map
	\label{def:mapping}%
	\textbf{写像}$f$とは,始集合$A$と終集合$B$との間の関係$R$であり,始集合の要素$a$について,一意的に$aRb$が決定できるものである.
\end{definition}
\begin{definition}[関数] % function
	\textbf{関数}$f$とは,終集合が数の集合であるような写像のことである.
\end{definition}
% % % 独立変数,従属変数
% % % cod(f) = R -> 実数値関数
% % % dom(f) = cod(f) = R -> 実関数
% % % cod(f) = C -> 複素数値関数
% % % dom(f) = cod(f) = C -> 複素関数
\begin{rem*}
	写像あるいは関数$f$は,始集合$A$と終集合$B$を伴って$f : A \rightarrow B$と表記される.また,始集合のことを\textbf{始域}と呼び$\dom f$と表記する.同様に,終集合のことを\textbf{終域}と呼び$\cod f$と表記する.これは,誤解を恐れずに言うならば方言のようなものである(発展するに当たって歴史的背景が異なる分野において,同じ意味を表す異なる単語があることは不思議ではない).
\end{rem*}
% % % dom(f)もcod(f)もfが主語であり,関数全体の集合でどうこうを考えるためか?
% % % 始集合,終集合は,いずれも集合と名前がついている通り集合論の話っぽい
% % % 始域,終域は,それに対して写像,つまりマッピングのイメージに近い
\begin{rem*}
	写像あるいは関数$f$について,特に要素の関係を強調する場合には,$f : a \mapsto b$と表記される.$f$は関係であるため,要素として順序対$(a,b)$を持つ.これを,$afb$と表記しても良いが,一般的に$f(a) = b$と表記する.この$a$に対して,$b$が$f$によって指定されることを,「$a$が$f$によって$b$に写される」といい,$b$のことを$a$における$f$の\textbf{値}と呼ぶ.
\end{rem*}
% % % 始集合,initial set, source
% % % 始域,domain
% % % 定義域,domain of definition
% % % 終集合,target set, target
% % % 終域,codomain
% % % 値域,range
% % % 像,image
% % % 始集合と始域,終集合と終域は,同じ意味の語彙である.対して,定義域は始域の部分集合として,像は終域の部分集合である.値域は像の言い換えである場合がほとんどだが,文献によっては終域の意味の場合もある.
% % % image <= range <= codomain
% % % また,domainと書かれている場合,ほとんど定義域の意味で書かれていることが多い.写像の場合は,始域と定義域は等しいので,ほとんど区別されない.
% % % 像については,元の像,部分集合の像,写像の像と3種類の意味がある.元の像については,値(value)と呼ばれることもある.
\begin{example*}
	関数$f : \mathbb{R} \rightarrow \mathbb{R}; a \mapsto a^2$について,始域と終域はともに実数全体の集合$\mathbb{R}$である.また,$2$における$f$の値は,$4$である.つまり,$f(2) = 4$である.
\end{example*}

\begin{definition}[像]
	写像あるいは関数$f$について,その\textbf{像}$\im f$とは,始域の要素に対する$f$の値を集めた集合のことである.像は,終域の部分集合である.
\end{definition}
\begin{rem*}
	意味合いが似てるものに,値域と呼ばれる語句があるが,ほとんどの場合,像と同じ意味である.また,定義域と呼ばれる語句は,本来,始域の部分集合であるが,写像の場合は,始域と定義域は等しいので,ほとんど同一視されている.
\end{rem*}
\begin{example*}
	関数$f : \mathbb{R} \rightarrow \mathbb{R}; a \mapsto a^2$について,像$\im f$は,正の実数全体の集合$\mathbb{R}_{\geq 0}$である.
\end{example*}

\newpage
\section{代数関数}
% % % 押さえたい内容
% % % n次関数,因数定理,因数分解,解と係数の関係,複素共役,平方完成,関数のグラフの平行移動,簡単なグラフの概形
% % % 閉区間・開区間と,2次不等式
代数関数とは,関数の中でも特に,始域の要素$a$と終域の要素$b$とを結びつける二項関係が,ある\textbf{代数式}(多項式や有理式,無理式の総称)によって定義される関数のことを指す.
代数式の例を,表\ref{table:algebraicExpression}に示す.ここで,$x$は,単なる記号であり,これを\textbf{不定元}と呼ぶ.
% % % 係数,項,項の次数,多項式の次数,モニック(単多項式),定数項
% % % 因数,既約多項式
% % % 変数,不定元
% % % 式の値,恒等式,方程式

\begin{table}[!h]
	\centering
	\caption{代数式}
	\label{table:algebraicExpression}
	\begin{tabular}{cccc|c}
		& & & & 例 \\
		\hline
		\multirow{6}{*}{代数式} & & \multirow{4}{*}{多項式} & 定数式 & $3$ \\
		& & & 一次式 & $2x+1$ \\
		& 有理式& & 二次式 & $3x^2+x+2$ \\
		& & & n次式 & $\sum_{i=0}^{n}a_ix^i\quad(a_n \neq 0)$ \\
		\cline{3-5}
		& & 分数式 & & $6/x, (2x+1)/(3x^2+x+2)$ \\
		\cline{2-5}
		& 無理式 & & & $\sqrt{x}, x^{2/3}$ \\
	\end{tabular}
\end{table}

\begin{comment}
\begin{table}[!h]
	\centering
	\caption{代数式}
	%\label{table:algebraicExpression}
	\[
		\text{代数式}
		\begin{cases}
			\text{有理式}
			\begin{cases}
				\text{多項式} 
				\begin{cases}
					\text{定数式} & 3, c\\
					\text{一次式} & ax+b \\
					\text{二次式} & ax^2+bx+c \\
					\text{$n$次式} & \sum_{i=0}^{n}a_ix^i
				\end{cases} \\
				\text{分数式} & \hspace*{-23mm}a/x
			\end{cases} \\
			\text{無理式} & \hspace*{-10mm}\sqrt{x}
		\end{cases}
	\]
\end{table}
\end{comment}

この節では,まず,多項式の因数定理を学ぶ,因数定理は,高次の多項式を,低次の多項式の積へと分解するために必要な因数を教えてくれる定理である.これから先,代数関数の積分を計算する際に,使用する定理なので,いくつかの計算練習も行う.
また,方程式や不等式について復習し.グラフとの関係について取り上げる.

\begin{comment}
\begin{definition}[二次式]
	$x$を不定元,$a, b, c$を定数とし,特に$a \neq 0$とする.このとき$ax^2+bx+c$で表現される多項式を,二次式と呼ぶ.
\end{definition}
\end{comment}
\begin{definition}[代入]
	代数式$P\langle x\rangle$に対し,不定元$x$を,異なる不定元あるいは集合$S$の要素$a$に置き換えることを\textbf{代入}と呼ぶ.$a$を代入することによって得られる$P\langle a\rangle$のことを,$a$において\textbf{評価}した値,あるいは\textbf{式の値}と呼ぶ.
\end{definition}
\begin{definition}[根]
	$P\langle x\rangle$を多項式とする.$P\langle x\rangle$に代入したとき,式の値が$0$となるような$a$のことを,多項式の\textbf{根}と呼ぶ.
\end{definition}
\begin{example*}
	二次式の根を求める方法の1つに,因数分解を利用する方法がある.代表的な二次式の因数分解の公式に
	\[
		x^2-(\alpha+\beta)x+\alpha\beta = (x-\alpha)(x-\beta)
	\]
	があり,このように因数分解できる場合,根は$\alpha, \beta$である.
\end{example*}
\begin{example*}
	二次式$x^2-5x+6$は,$(x-2)(x-3)$と因数分解できるため,この二次式の根は,$2, 3$である.また,二次式$x^2+2x+1$は,$(x+1)^2$と因数分解できる.これは,$(x+1)(x+1)$をまとめて表記したものであるため,$x^2+2x+1$の根は,$-1, -1$である.ただし,いくつかの根のうち等しい値がある場合,それらは等しい値ごとにまとめて書くため,この場合は$-1\,\text{(\textbf{重根})}$と表記される.
\end{example*}
\begin{theorem}[因数定理]
	多項式$P\langle x\rangle$の根の1つを$a$と置く.このとき,$P\langle x\rangle$は$(x-a)$を因数に持ち,$P\langle x\rangle = (x-a)Q\langle x\rangle$と表現することができる.ここで$Q\langle x\rangle$は,$P\langle x\rangle$の次数より1つ低い次数を持つ多項式である.
\end{theorem}
\begin{rem*}
	この定理の主張は,ある多項式$P\langle x\rangle$について,$a$を代入したとき,式の値$P\langle a\rangle$が$0$になるならば,$P\langle x\rangle$は$(x-a)$で割り切れる,ということである.従って,手計算の際には,$0, 1, -1$など小さい値から順に代入していき,式の値が$0$になった数を使って,多項式の割り算を行うという手順を踏むことになる.
\end{rem*}
\begin{example*}
	多項式$P\langle x\rangle = x^3+x^2-x-1$は,$x = 1$を代入すると,$P\langle 1\rangle = 0$となる.つまり,この多項式の根の1つが$1$であることが分かる.よって,多項式$P\langle x\rangle$は,$(x-1)$を因数に持つため,これで割ることができ,$P\langle x\rangle = (x-1)(x^2+2x+1)$となる.したがって,多項式$P\langle x\rangle$を因数分解すると,$(x-1)(x+1)^2$となり,$P\langle x\rangle$の根は,$1, -1\,\text{(重根)}$であることが分かる.
\end{example*}
% % % 練習問題を別途作成する.

\begin{definition}[恒等式]
	$P\langle x\rangle, Q\langle x\rangle$を代数式とする.集合$S$における,すべての要素$a$に対して,$P\langle a\rangle = Q\langle a\rangle$が成り立つとき,これを\textbf{恒等式}と呼ぶ.
\end{definition}
\begin{definition}[方程式]
	$P\langle x\rangle, Q\langle x\rangle$を代数式とする.集合$S$における,ある要素$a$に対して,$P\langle a\rangle = Q\langle a\rangle$が成り立つかどうか真偽が決定できるとき,これを\textbf{方程式}と呼ぶ.また,成り立つような$a$のことを\textbf{解}と呼ぶ.
\end{definition}
\begin{definition}[不等式]
	$P\langle x\rangle, Q\langle x\rangle$を代数式とする.また演算$\star$を$<, \leq, >, \geq$のいずれかと決める.集合$S$における,ある要素$a$に対して,$P\langle a\rangle \star Q\langle a\rangle$が成り立つかどうか真偽が決定できるとき,これを\textbf{不等式}と呼ぶ.また,成り立つような$a$のことを\textbf{解}と呼ぶ.
\end{definition}
\begin{rem*}
	方程式や不等式は,解が複数あることがほとんどである.そのため,すべての解を要素として持つ集合を\textbf{解集合}と呼び,解集合を求めることを,方程式あるいは不等式を\textbf{解く},という.また,解集合のことを,そのまま解と呼ぶこともある.
\end{rem*}

\begin{definition}[多項式関数のグラフ]
	\label{def:graphPolynomial}%
	ある多項式$P\langle X\rangle$が与えられているとき,関数$f : \mathbb{R} \rightarrow \mathbb{R}; x \mapsto P\langle x\rangle$の\textbf{グラフ}とは,$f$の要素である順序対$(x, P\langle x\rangle)$を,$xy$平面上の点として描いたものである.
	このとき$x$は,始域$\mathbb{R}$の全ての要素となり得るので,\textbf{変数}あるいは未知数と呼ばれる.
	通常は,$x$における$f$の値を$y$と置き,$y = f(x)$と表記する.$x$における$f$の値は,$P\langle x\rangle$であるため,これを用いて$y = P\langle x\rangle$や$f(x) = P\langle x\rangle$と表記することもある.
\end{definition}
% % % 分数関数は,分母のゼロ割りが発生しないように定義域を変更する必要があり,無理関数も,根号の中身が負にならないように,定義域を変更する必要がある.
\begin{rem*}
	多項式$P\langle x\rangle$と多項式関数$f(x)$の違いは,単なる記号である$x$か,何らかの要素を取り得る変数としての$x$かの違いである.そのため,定義\ref{def:graphPolynomial}では,単なる記号であることを強調するため,多項式の不定元に$X$を用いた.
\end{rem*}
% % % 変数は,要素が固定である定数に対する,用語であり,要素が変化することが期待されている.
% % % 変数が取り得る要素の集合を,変域あるいは定義域と呼ぶ.
% % % 関数の文脈において,変数とは「関数の引数となるべき何らかの集合の元を代表するもの」である.
\begin{definition}[閉区間・開区間]
	ある実数$a, b$について(ただし$a \leq b$),\textbf{閉区間}$\left[a, b\right]$とは,$\set{x \mid a \leq x \leq b}$となる集合のことである.また,\textbf{開区間}$\left(a, b\right)$とは,$\set{x \mid a < x < b}$となる集合のことである.不等号の組み合わせによって,$\left[a, b\right) = \set{x \mid a \leq x < b}$と$\left(a, b\right] = \set{x \mid a < x \leq b}$も定義される.
\end{definition}
\begin{rem*}
	$\left(a, b\right)$と書かれている場合,文脈から開区間なのか順序対なのか判断する必要がある.また,閉区間・開区間については,数直線上に描くことができ,図\ref{fig:interval}のようになる.特に,ある実数$c$よりも小さい数の集合を,$\left(-\infty, c\right)$と表記したり,ある実数$c$以上の数の集合を$\left[c, \infty\right)$と表記したりする.
	
	\vfill
	\begin{figure}[!h]
		\caption{閉区間と開区間の図示}
		\label{fig:interval}
	\end{figure}
\end{rem*}
% % % \infty は通常,実数には含まれないが,ここではあえて触れないことにする.
\clearpage

\begin{example*}
	多項式$x-2$と$2x-3$について,$x = 1$を代入すると,お互いの式の値が$-1$となり等しくなる.つまり,方程式$x-2 = 2x-3$の解は,$x = 1$である.これは,一次関数$y = x-2$と$y = 2x-3$の交点$(1, -1)$に対応する.
\end{example*}
\begin{example*}
	方程式$x^2-5x+6 = x-2$の解は,$x = 2, 4$である.これは,二次関数$y = x^2-5x+6$と一次関数$y = x-2$の交点$(2, 0), (4, 2)$に対応する.
\end{example*}
\begin{example*}
	不等式$x^2-5x+6 \leq x-2$の解は,$2 \leq x \leq 4$である.これは,二次関数$y = x^2-5x+6$よりも一次関数$y = x-2$のほうが大きくなるか等しくなる$x$の区間が,$\left[2, 4\right]$であることに対応する.同様に,不等式$x^2-5x+6 > x-2$の解は,$x < 2, 4 < x$である.これは,二次関数$y = x^2-5x+6$よりも一次関数$y = x-2$のほうが小さくなる$x$の区間が,$\left(-\infty, 2\right) \cup \left(4, \infty\right)$であることに対応する.
\end{example*}

ここまでの例について,図示したものが図\ref{fig:equation}である.

\vfill
\begin{figure}[!h]
	\caption{方程式の解,不等式の解とグラフとの関係の図示}
	\label{fig:equation}
\end{figure}
\clearpage

% % % 解と係数の関係,複素共役 -> 計算上のテクニックになるので,とりあえず一旦保留.というか代数学の話では??
% % % 平方完成,関数のグラフの平行移動,簡単なグラフの概形
この節の終わりに,二次関数のグラフを書くために必要な手法である平方完成を学ぶ.
\begin{definition}[平方完成]
	二次式$ax^2+bx+c$について,$a(x-p)^2+q$の形に変形することを\textbf{平方完成}と呼ぶ.
\end{definition}
% % % 一般形 : ax^2+bx+c
% % % 標準形 : a(x-p)^2+q
% % % (因数)分解形 : a(x-s)(x-t)
\begin{algorithm}
	\begin{align}
		ax^2+bx+c &= a\left(x^2+\frac{b}{a}x\right)+c \tag{1} \\
		&= a\left\{\left(x+\frac{b}{2a}\right)^2-\frac{b^2}{4a^2}\right\}+c \tag{2} \\
		&= a\left(x+\frac{b}{2a}\right)^2-\frac{b^2}{4a}+c \tag{3} \\
		&= a\left(x+\frac{b}{2a}\right)^2-\frac{b^2-4ac}{4a} \tag{4}
	\end{align}
	
	各計算ステップは,以下の通りである.
	\begin{enumerate}[(1)]
		\item $ax^2+bx$について$a$でくくる.
		\item $x^2+\frac{b}{a}x$を変形する.ここでは,$\left(x+\frac{b}{2a}\right)^2$を展開することによって,$x^2+\frac{b}{a}x+\frac{b^2}{4a^2}$となるので,余分な$\frac{b^2}{4a^2}$を引くことによって,$x^2+\frac{b}{a}x$を作っている.
		\item $a$を$\left(x+\frac{b}{2a}\right)^2$と$-\frac{b^2}{4a^2}$に,分配する.
		\item $-\frac{b^2}{4a}$と$c$を通分する.
	\end{enumerate}

	結果として
	\begin{empheq}[left=\empheqlbrace]{align*}
		p &= -\frac{b}{2a} \\
		q &= -\frac{b^2-4ac}{4a}
	\end{empheq}
	を得る.
\end{algorithm}
\begin{rem*}
	二次関数$y = f(x) = a(x-p)^2+q$は,$x = p$のとき,$a$の符号によって,最大値あるいは最小値として$y = q$をとる.この,二次関数の最大値あるいは最小値を決める点$(p, q)$を,\textbf{頂点}と呼ぶ.
\end{rem*}
\begin{example*}
	二次式$x^2-5x+6$は,$\left(x-\frac{5}{2}\right)^2-\frac{1}{4}$と平方完成できるため,二次関数$f(x) = x^2-5x+6$は,頂点$\left(\frac{5}{2}, -\frac{1}{4}\right)$を通るグラフである.また,この関数の像$\im f$は,区間$\left[-\frac{1}{4}, \infty\right)$である.
\end{example*}

\begin{definition}[グラフの平行移動]
	多項式関数$f(x)$が与えられているとき,新たに多項式関数$g(x) = f(x-p) + q$を作ることができる.このとき,$g(x)$のグラフは,$f(x)$のグラフを$x$軸方向に$p$,$y$軸方向に$q$だけ動かしたものであり,これをグラフの\textbf{平行移動}という.
\end{definition}
\begin{example*}
	関数$f(x) = x^2$とすると,関数$g(x) = x^2-5x+6$は,$g(x) = \left(x-\frac{5}{2}\right)^2-\frac{1}{4} = f\left(x-\frac{5}{2}\right)-\frac{1}{4}$と書くことができる.したがって,$g(x) = x^2-5x+6$のグラフは,$f(x) = x^2$のグラフを,$x$軸方向に$\frac{5}{2}$,$y$軸方向に$-\frac{1}{4}$だけ動かしたものである.
\end{example*}

ここまでの例について,図示したものが図\ref{fig:quadraticFunction}である.

\vfill
\begin{figure}[!h]
	\centering
	\begin{tikzpicture}
		\draw[help lines, step=0.25] (0,0) grid (10,10);
		\draw (5,5) node[below left]{O};
		\draw[thick, ->] (-0.5,5)--(10.5,5) node[below=3.0] {$x$};
		\draw[thick, ->] (5,-0.5)--(5,10.5) node[right=3.0] {$y$};
	\end{tikzpicture}
	\caption{二次関数のグラフの図示}
	\label{fig:quadraticFunction}
\end{figure}
\clearpage

\section{指数関数}
	% % % End of Body

	% % % Begin of Reference
	\bibliographystyle{IEEEtran}
	\nocite{*}
	\bibliography{ref}
	%\addcontentsline{toc}{chapter}{References}
	% % % End of Reference

	% % % Begin of index
	%\printindex
	% % % End of index

	% % % Begin of appendix
	%\appendix
	%\include{./Chapter/Ap01}
	% % % End of appendix
\end{document}
