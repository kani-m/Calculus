% % % Document Type
% http://www.biwako.shiga-u.ac.jp/sensei/kumazawa/tex/book.html
% http://ichiro-maruta.blogspot.jp/2013/03/latex.html
\RequirePackage[l2tabu, orthodox]{nag}
\documentclass[12pt, a4j, openany, uplatex, dvipdfmx]{jsbook}

% % % Paper size
%\usepackage[top=1truein, bottom=1truein, left=1.5truein, right=1truein]{geometry} % 用紙サイズの設定のため

% % % Necessary
\usepackage{amssymb, amsmath, amsthm, latexsym} % For mathematics [Before hyperref]
\usepackage[dvipdfmx,setpagesize=false]{hyperref} % For inserting hyperlink [Before graphicx]
\usepackage{pxjahyper} % https://texwiki.texjp.org/?hyperref
\usepackage[dvipdfmx]{graphicx} % For inserting figure
\usepackage[labelformat=simple]{subcaption} % For caption of figures and tables
%\renewcommand\thesubfigure{(\alph{subfigure})}
%\renewcommand\thesubtable{(\alph{subtable})}
\usepackage[usenames,dvipdfmx]{color} % For coloring
\usepackage[shortlabels]{enumitem} % For useful enumerate and itemize environment
\usepackage{comment} % For commenting multiline
\usepackage{cite} % For citing references [After hyperref]
\usepackage{url} % For inserting URL
\usepackage{acro} % For listing abbreviations and symbols
\usepackage{makeidx} % For making index

% % % Convenient for graph
%\usepackage{tikz-cd} %load package after color.sty & soul.sty

% % % Convenient for tables and emphasis
\usepackage{tabularx} % For flexible table
\usepackage{multirow} % For combining multi rows for table
%\usepackage{colortbl} % For coloring table
\usepackage{longtable} % For long table across multiple pages
%\usepackage[normalem]{ulem} % For flexible underline etc.
%\usepackage{ascmac} % For framing multiline -> bxascmac に変更したほうがよい?

% % % 文書の見た目のためのパッケージ
\usepackage{setspace} % For single space or double space
\usepackage{afterpage} % For inseting newpage 
\usepackage{datetime} % For date style of title
\usepackage{fancyhdr} % For header and footer
% % tocloft を include すると目次の番号と文章が被る?
%\usepackage{tocloft} % 図目次、表目次の見た目を変えるため

% % % Convenient for document of science and technology
\usepackage{numprint} % 数値の整形
\usepackage{siunitx} % For using SI units
\usepackage{listings} % For inserting programming code

% % % Option
\usepackage{docmute} % For compilation of divided files
\usepackage{bxcoloremoji} % For using emoji

% % % Reference for packages
% % geometry.styについて
% http://joker.hatenablog.com/entry/2012/07/09/153537
% % tocloft.styについて
% http://www.biwako.shiga-u.ac.jp/sensei/kumazawa/tex/tocloft.html
% http://tex.stackexchange.com/questions/20337/adding-word-table-before-each-entry-in-list-of-tables
% % makeidx.sty
% http://www.biwako.shiga-u.ac.jp/sensei/kumazawa/tex/makeidx.html
% % %

% % % Config for hyperref
\hypersetup{ %
	breaklinks=true, %
	colorlinks=false, %
	urlcolor=blue, %
	urlbordercolor={0 1 1}}

% % % Config for header of chapter
% % http://oversleptabit.com/?p=390
\makeatletter
\def\@makechapterhead#1{%
	\vbox to 50mm{\kern7mm\parindent\z@
		\raggedright
		\reset@font
		\huge\bfseries
		\ifnum \c@secnumdepth >\m@ne
		\hskip10mm{\fontsize{20Q}{20Q}\selectfont\@chapapp
			{\Huge\thechapter}\@chappos}
		\vskip1.5mm
		\hrule \@height 0.1mm
		\vskip.5mm
		\hrule \@height 0.5mm
		\vskip2mm
		\begin{flushright}
			\fontsize{28Q}{29Q}\bfseries\selectfont#1\par
		\end{flushright}
		\vfill
		\fi}
	\nobreak
}
\makeatother

% % % Define Command
\newcommand{\bm}[1]{\mbox{\boldmath $#1$}}
\newcommand{\dif}{\mathrm{d}}

% % % Re-define names

% % % Reference: 
% http://tex.stackexchange.com/questions/86666/how-to-create-both-list-of-abbreviations-and-list-of-nomenclature-using-nomencl
% ftp://ftp.kddilabs.jp/CTAN/macros/latex/contrib/acro/acro_en.pdf
% % %

% Configuration for hyperlink and list of acronyms
\acsetup{hyperref=true, list-style=longtable, list-heading=chapter*}

% class `abbrev': abbreviations:
\DeclareAcronym{vs}{
  short = VS ,
  long  = Vector Space ,
  class = abbrev
}


% class `nomencl': nomenclature (symbol)
\DeclareAcronym{M}{
  short = $\bm{M}$ ,
  long  = 行列 ,
  sort  = A ,
  class = nomencl
}
\DeclareAcronym{v}{
  short = $\bm{v}$ ,
  long  = ベクトル ,
  sort  = A ,
  class = nomencl
}
\DeclareAcronym{R}{
  short = $\mathbb{R}$ ,
  long  = 実数 ,
  sort  = R ,
  class = nomencl
}

 % List of abbreviations and symbols
\makeindex
\begin{document}
	% % % Begin of Front matter
	% 表題、目次、序文等からなるフロントマター部用の設定をする
	\frontmatter
	% 表紙
	\title{微積分}
	\author{Yuki M.}
	\date{\usdate\today}
	\maketitle
	% 目次
	%\setcounter{tocdepth}{3}
	\tableofcontents
	% 図、表、略語、記号一覧
	%\listoffigures
	%\listoftables
	%\printacronyms[include-classes=abbrev,name=List of Abbreviations]
	%\printacronyms[include-classes=nomencl,name=List of Symbols]
	% 略語一覧も記号一覧も表示しない場合,参照を切るために以下を設定する
	\acsetup{hyperref=false}
	% % % End of Front matter

	% % % Begin of Body
	\mainmatter
	% Chapterごとにファイルを分けて,それぞれをincludeする
	\chapter{はじめに}
この資料は,微積分学の講義のための参考資料です.自分の備忘録も兼ねているのでやや冗長な表現もあるやもしれません.
筆者もまだまだ勉強中であるため,理解が追いつかず根本的に誤解している箇所があるかと思います.
ぜひとも,「ここの文章はおかしいのではないか?」等のご指摘をお待ちしております.
また,誤字・脱字,および論理的な誤りがあればご一報ください.

\section{微積分とは?}
微積分は線形代数と並び現代数学の基礎として,大学など高等教育機関において広く学ばれている教養科目である.
また,線形代数と同様に,自然科学・工学をはじめとした応用科学・社会科学など幅広い分野で応用されている計算ツールである.

そもそも微積分とは,簡単に言ってしまえば「変化」を捉えるための数学である.世の中の物理量(例えば,物体の速度や温度など)のほとんどは,位置や時間の変化に伴って変化する.その変化の度合いが,微かな変化の中にどれぐらいであったのかを,捉える数学が「微分」であり,変化の度合いの積み重ねから異なる位置・時間における物理量を予測する数学が「積分」である.

% % % 歴史的背景や具体的な応用例など,導入部を厚くしたい
% 歴史的には,17世紀にニュートンが作り出したと言われており,微積分の概念は,その後の数学に大きな影響を与えたとされている.

\section{講義の内容}
この講義においては,まず始めに表\ref{table:ele_func}に示すような初等関数を学ぶ.
これは「変化」を捉える数学である「微積分」を学ぶ上で,大きな寄り道に思えるかもしれない.実際のところ「微分」「積分」の概念を理解することと,初等関数を理解することは,全く別の話である.しかし,これらの初等関数は,様々な現象を数式にモデル化する上で使用されるものであり,実用上において微積分と初等関数は切っても切れない関係にあることを理解してほしいため,まず始めに取り上げるものとする.

\begin{table}[!h]
	\centering
	\caption{初等関数}
	\label{table:ele_func}
	\begin{tabular}{c|c}
		& 例 \\
		\hline
		代数関数 & $x^2 + x - 2$ \\
		指数関数 & $2^x$ \\
		対数関数 & $\log_2{x}$ \\
		三角関数 & $\sin x, \cos x, \tan x$ \\
		逆三角関数 & $\arcsin x, \arccos x, \arctan x$ \\
	\end{tabular}
\end{table}

次に,微積分を学ぶ上で核となる「変化」を明確に記述するために極限と呼ばれる概念を取り上げる.つまり,微分とは「微かな変化」を捉えるための数学のことであったが,「``微かな''変化」とは一体どのようなものなのか,微小量と呼ばれる考え方を導入し,微分が持つ本質的な意味を探っていく.

その後,定積分と呼ばれる微小量の積み重ねによって表される計量が,微分の逆演算としての不定積分を用いて計算した結果と等しくなる微分積分学の基本定理について取り上げ,積分が表す2通りの意味を理解し,先人たちが築き上げた大きな金字塔を追体験していく.

またこの講義では,微分・積分の概念の意味を理解すると同時に,初等関数の基本的な微分・積分の計算を行っていく.
具体的には,様々な初等関数のグラフの概形や極値,様々な図形の面積・体積を求めるといった問題を想定している.

その上で,実用上欠かすことができない微分方程式や級数について触れ,もし時間があれば,多変数の微分・積分の初歩的概念について取り上げることとする.

\section{到達目標}
この講義では,以下に示す項目を達成することを目的とする.
\begin{enumerate}[(1)]
	\item 微積分が持つ本質的な意味をつかむ.
	\item 微積分に関する基本的な計算ができるようになる.
	\item 微積分の知識を使い,初等関数のグラフの概形や基本的な図形の計量が求められるようになる.
	\item 単純な微分方程式が解けるようになる.
	\item[(*5)] 各定理の主張が理解でき,その証明が追えるようになる.
\end{enumerate}

(1)(2)については,受講者全員に達成してほしい目標として設定した.(3)(4)については,講義全体を通して身につけて欲しい計算力に関する目標である.(*5)については,高度な目的となるため,深くは言及しないが,論理的な思考力を身につけるため,ぜひチャレンジして欲しい目標である.
%微積分は,統計学をはじめとした様々な学問分野で欠かせない計算ツールであるため,毎回の演習を通して理解の深化と計算力の向上を促したいと考えている.

\section{評価基準}
表\ref{table:criteria}にこの講義の評価基準を示す.この内,筆記試験については,中間試験を実施するかどうか履修人数が確定したタイミングで話し合いたいと考えている.また,実施する場合は,中間試験と期末試験の割合の比率についても同様に話し合いたい.

\begin{table}[!h]
	\centering
	\caption{評価項目とその割合}
	\label{table:criteria}
	\begin{tabular}{c|c}
		評価項目 & 割合 \\
		\hline
		講義への出席 & $\SI{30}{\percent}$ \\
		毎回の課題 & $\SI{30}{\percent}$ \\
		筆記試験 & $\SI{40}{\percent}$ \\
	\end{tabular}
\end{table}

\section{日程と進め方}
表\ref{table:schedule}にこの講義の日程を示す.原則として毎週水曜日と金曜日の6限(17:40 -- 19:10)に講義を行う.進め方としては,1コマをさらに2つに分割して,45分間を1つのブロックとし,新しい概念の説明と,それに付随する演習問題の解説を当てる予定である.
%高等学校における講義と演習が時間的に密接にある授業を目指していきたいと考えている.
毎回の課題については,小テストにするか宿題形式にするか現時点(2017/9/28)では決定していない.宿題形式の場合は,別途Webサービスを通して提出という形を取っていきたい.

また,仮に中間試験を行うとすれば,11月下旬を想定している.

\begin{table}[!h]
	\centering
	\caption{講義の日程とテーマ}
	\label{table:schedule}
	\begin{tabular}{c|c}
		日付 & 内容 \\
		\hline
		10/04 & \multirow{4}{*}{初等関数の復習} \\
		10/06 & \\
		10/11 & \\
		10/13 & \\
		\hline
		10/18 & \multirow{2}{*}{極限と導関数} \\
		10/20 & \\
		\hline
		10/25 & \multirow{4}{*}{初等関数の微分} \\
		10/27 & \\
		11/01 & \\
		11/08 & \\
		\hline
		11/10 & \multirow{2}{*}{高階導関数} \\
		11/15 & \\
		\hline
		11/17 & \multirow{2}{*}{初等関数のグラフの概形と極大値・極小値} \\
		11/24 & \\
		\hline
		11/29 & \multirow{2}{*}{定積分と原始関数} \\
		12/01 & \\
		\hline
		12/06 & \multirow{4}{*}{初等関数の積分} \\
		12/08 & \\
		12/13 & \\
		12/15 & \\
		\hline
		12/20 & \multirow{2}{*}{図形の面積・体積と広義積分} \\
		12/22 & \\
		\hline
		12/27 & \multirow{2}{*}{微分方程式} \\
		01/05 & \\
		\hline
		01/10 & \multirow{2}{*}{級数} \\
		01/12 & \\
		\hline
		01/17 & \multirow{2}{*}{偏微分} \\
		01/19 & \\
		\hline
		01/24 & \multirow{2}{*}{重積分} \\
		01/26 & \\
	\end{tabular}
\end{table}

	% % % End of Body

	% % % Begin of Reference
	\bibliographystyle{IEEEtran}
	\nocite{*}
	\bibliography{ref}
	%\addcontentsline{toc}{chapter}{References}
	% % % End of Reference

	% % % Begin of index
	%\printindex
	% % % End of index

	% % % Begin of appendix
	%\appendix
	%\include{./Chapter/Ap01}
	% % % End of appendix
\end{document}
